\section{\toolname Evolution}\label{sec:evolution}

Currently, \toolname is composed by a coverage analysis testing
tool, a slicing too, and a static metrics tool for Java programs.
The tool uses the Java bytecode (.class files) instead of the Java
source code to collect the information necessary to perform its
activities. This characteristics allows the tool to be used to
test any kind of Java program, including Java components.

Since most part of the testing criteria currently developed for
component based testing are functional testing criteria, by
implementing a set of structural testing criteria applied directly
in the Java bytecode, \toolname enables the structural testing of
Java components and also the identification of which part of such
components needs more test or has not yet been covered.

Even when no source code is available, the tester can at least
identify which part of the bytecode requires more test, or, in
case of a fault is identified, which part of the component is more
probable to contain that fault. This information can be given to
the component provider such that the component user can receive
another corrected version of the component.

Improvements forth coming of \toolname are:

\begin{itemize}
    \item Development of additional testing criteria to deal not
    only with intra-method testing but also with inter-method and
    inter-class testing;

    \item Development of additional heuristic to improve the slicing
    tool, such that a smart debugging and a ease fault localization to
    be possible. We intent to investigate different slicing criteria,
    including the ones that consider also data-flow information, to implement
    on \toolname different heuristics to help the tester on fault localization
    and smarting debug;

    \item Development of a set of complexity metrics to help the definition
    of a incremental testing strategy considering the complete hierarchy of
    the classes under testing;

    \item Evaluation, development and implementation of heuristics
    to automatically identify part of the infeasible testing
    requirements;

    \item Implementation of algorithms to optimize the number of
    testing requirements to be evaluated, using the
    essential-branches' concept;

    \item Improvement in the graphical interface to ease the
    testing activity as much as possible.
\end{itemize}

Moreover, to show the benefices of the tool, experiments will be
carried out exploring the different aspects of \toolname.
